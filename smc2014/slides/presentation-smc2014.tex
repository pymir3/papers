\documentclass{beamer}
\RequirePackage[utf8]{inputenc}
%\RequirePackage{textcomp}
\usepackage[english]{babel}    % hiphenao em portugues
\usepackage[T1]{fontenc} 
\usepackage{ae} 
\usepackage{amsmath}
\usepackage{tikz}
\usepackage{pgfplots, pgfplotstable}
%%%% TIKZ STUFF %%%%%
\usetikzlibrary{calc,trees,positioning,arrows,chains,shapes.geometric,%
    decorations.pathreplacing,decorations.pathmorphing,shapes,%
    matrix,shapes.symbols}


% Define block styles
\tikzstyle{block} = [rectangle, draw, fill=white!100, 
    text width=8em, text centered, rounded corners, minimum
    height=2em]
\tikzstyle{block2} = [rectangle, draw, fill=gray!20, 
    text width=8em, text centered, rounded corners, minimum height=2em]
\tikzstyle{line} = [draw, -latex']

\usepackage{beamerthemedefault}
%\RequirePackage{floatflt}
\usepackage{subfigure}

\title{Query-by-Multiple-Examples: Content-Based Search in Computer-Assisted Sound-Based Musical Composition}
\author{Tiago Fernandes Tavares, Jônatas Manzolli\\Interdisciplinay
  Nucleus of Sound Communication\\University of Campinas - Brazil\\ \url{tiagoft@gmail.com}\\
  \url{http://www.nics.unicamp.br/~tiago}}
\date{Sept/17/2014}

\begin{document}

\frame{\titlepage}

\section[Outline]{}

%\frame{\tableofcontents}


\section{Introduction}
\frame{
	\frametitle{Sound-based Music}
\Large
Musical discourse is based on the evolution of sonic
characteristics.\\
Examples: Electroacoustic, Acousmatic, Mixed Music.\\
Techniques: sequencing, granular synthesis, other digital effects...\\
}

\frame{
	\frametitle{Problem: search for content}
\Large
Musicians might...
\begin{enumerate}
\item Record their own sound database
\item Privately share their database
\item Search the Internet for more sounds
\end{enumerate}
Problem: efficiently acquire new sounds for a database!
}

\frame{
	\frametitle{Solution 1: TAGS}
\Large
Problems with this solution:
\begin{itemize}
\item We disagree on the meaning of tags (what is ``bright''?)
\item We disagree on what tags are suitable for each sound
\end{itemize}
}

\frame{
  \frametitle{Solution 2: Query-by-example}
\Large
\begin{enumerate}
\item Map the audio file to some $\mathbb{R}^N$ vector space using
  objetive features calculated from the signal (e.g., mean and
  variance of energy, spectral flux, spectral centroid...),
\item Perceptually similar sounds should be mapped to
  close points in $\mathbb{R}^N$.
\end{enumerate}
Problem with this solution: $\mathbb{R}^N$ lacks meaning!
}

\frame{
	\frametitle{Query by Multiple Examples}
\Large
Idea: a query is built using many sounds that share the desired characteristic.
\begin{itemize}
	\item Map sounds to $\mathbb{R}^N$
	\item Estimate a model from the elements of the query
  \item Check the database for elements that suit the model
\end{itemize}
}

\frame{
	\frametitle{Query by Multiple Examples}
\begin{figure}[ht!]
\centering
\begin{tikzpicture}[node distance = 4cm, auto]
    % Place nodes
    \node [block] (audio) {Audio Dataset};
    \node [block2, below of=audio, yshift=2.7cm] (composer) {Composer};
    \node [block, below of=composer, yshift=2.7cm] (perception) {Perception};
%    \node [block, below of=perception, yshift=2.7cm] (query) {Examples};

    \node [block2, right of=composer] (computer) {Computer};
    \node [block, below of=computer, yshift=2.7cm] (features) {Features};
    \node [block, below of=features, yshift=2.7cm] (model) {Model};
    \node [block, below of=model, yshift=2.7cm] (recommendation) {Recommendation};

    % Draw edges
    \path [line] (audio) -- (composer);
    \path [line] (audio) -| (computer);
    \path [line] (composer) -- (perception);
%    \path [line] (perception) -- (query);
    \path [line] (perception) |- node [xshift=1cm] {Query $\{e_j\}$}  (model);
    \path [line] (computer) -- (features);
    \path [line] (features) -- node {$\{d_i\} \in \mathbb{R}^N$} (model);
    \path [line] (model) -- (recommendation);
    \path [line,dashed] (recommendation) -| ([xshift=-0.5cm] composer.west) -- (composer);
\end{tikzpicture}
\label{fig:block-diagram}
\caption{Block diagram for the proposed system.}
\end{figure}
}

\frame{
  \frametitle{How to search using multiple examples?}
  \Large
We should find elements that are...
\begin{itemize}
\item <1-> Close to any element of the query (Minimum
  Distance)?
\item <2-> Close to any element of the query AND far from other
  elements in the database (Minimum Mutual Distance)?
\end{itemize}
These correspond to many single-example, K-NN-based queries!
}

\frame{
  \frametitle{How to search using multiple examples?}
  \Large
We can build a model!
\begin{itemize}
\item <1-> Build a generative model (Naive Bayes)
\item <2-> Before that, find directions in which the query spans and
  perform dimension reduction (PCA-Bayes)
\end{itemize}
These incorporate joint knowledge about the query set!
}

\frame{
  \frametitle{Experiments}
  \Large
\begin{itemize}
\item We gathered a small dataset (51 sounds)
\item 3 composers (C1, C2, C3) made sound groups as they thought it
  would fit
\item We used part of the group as a query and verified if the system
  would extrapolate the rest of the group (3 best results)
\end{itemize}
}

\frame{
  \frametitle{Results}
\begin{figure}[ht!]
\centering
%\caption{Results using Acc2.} 
\begin{tikzpicture}
\pgfplotstableread[row sep=crcr]{ % Read the data into a table macro
Label   MD  MMD  NB PB  \\
C1      17   17   17  21 \\
C2      17   21   23  22\\
C3      27   34   33   40 \\
}\datatable

\begin{axis}[
  width=1.0\textwidth,
  height=4.cm,
	xtick=data,
  ymin=0,
  ymax=100,
  ylabel=Hits (\%),
	xlabel=Composers,
	enlarge x limits = 0.20,
	legend style={legend columns=-1},
	ybar,
	bar width=10pt,
  xticklabels from table={\datatable}{Label}
]

\addplot [fill=white] table [x expr=\coordindex, y=MD] {\datatable};    % Plot the "First" column against the data index
\addplot [fill=gray!30]table [x expr=\coordindex, y=MMD] {\datatable};
\addplot [fill=gray!90] table [x expr=\coordindex, y=NB] {\datatable};
\addplot [fill=black] table [x expr=\coordindex, y=PB] {\datatable};
% \addplot  
% 	coordinates {(MD,0.38) (MMD,0.44)
% 		 (NB,0.42) (PB,0.52)};

% \addplot 
% 	coordinates {(MD,0.42) (MMD,0.51)
% 		 (NB,0.51) (PB,0.43)};

% \addplot 
% 	coordinates {(MD,0.61) (MMD,0.72)
% 		 (NB,0.67) (PB,0.76)};


\legend{MD,MMD,NB,PB}
\end{axis}
\end{tikzpicture}
%\label{fig:res_acc2}
\end{figure}
\begin{figure}[ht!]
\centering
%\caption{Results using Acc1.} 
\begin{tikzpicture}
\pgfplotstableread[row sep=crcr]{ % Read the data into a table macro
Label   MD  MMD  NB PB \\ 
C1      38   44   42  52 \\
C2      42   51   51  43\\
C3      61   72   67  76\\
}\datatable

\begin{axis}[
  width=1.0\textwidth,
  height=4.cm,
	xtick=data,
  ymin=0,
  ymax=100,
  ylabel=Res. with hits (\%),
	xlabel=Composers,
	enlarge x limits = 0.20,
	legend style={legend columns=-1},
	ybar,
	bar width=10pt,
  xticklabels from table={\datatable}{Label}
]

\addplot [fill=white] table [x expr=\coordindex, y=MD] {\datatable};    % Plot the "First" column against the data index
\addplot [fill=gray!30]table [x expr=\coordindex, y=MMD] {\datatable};
\addplot [fill=gray!90] table [x expr=\coordindex, y=NB] {\datatable};
\addplot [fill=black] table [x expr=\coordindex, y=PB] {\datatable};
% \addplot 
% 	coordinates {(MD,0.38) (MMD,0.44)
% 		 (NB,0.42) (PB,0.52)};

% \addplot 
% 	coordinates {(MD,0.42) (MMD,0.51)
% 		 (NB,0.51) (PB,0.43)};

% \addplot 
% 	coordinates {(MD,0.61) (MMD,0.72)
% 		 (NB,0.67) (PB,0.76)};


\legend{MD,MMD,NB,PB}
\end{axis}
\end{tikzpicture}
%\label{fig:res_acc1}
\end{figure}
}

\frame{
  \frametitle{Future work}
  \Large
  Some questions to be asked:
  \begin{itemize}
  \item Even if a result is unexpected, is it still useful?
  \item How to make the system easy to use?
  \item How will it behave in a large database?
  \item What other techniques will improve it?
  \end{itemize}
}

\frame{
  \frametitle{Thank you}
  \Large
  Tiago Fernandes Tavares\\
  \url{tiagoft@gmail.com}\\
  \url{http://www.nics.unicamp.br/~tiago}
}

\end{document}

