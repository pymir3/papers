% -----------------------------------------------
% Template for ICMC SMC 2014
% adapted and corrected from the template for SMC 2013,  which was adapted from that of  SMC 2012, which was adapted from that of SMC 2011
% -----------------------------------------------

\documentclass{article}
\usepackage{icmcsmc2014}
\usepackage{times}
\usepackage{ifpdf}
\usepackage[english]{babel}
\usepackage{tikz}
\usepackage{pgfplots, pgfplotstable}
%%%% TIKZ STUFF %%%%%
\usetikzlibrary{calc,trees,positioning,arrows,chains,shapes.geometric,%
    decorations.pathreplacing,decorations.pathmorphing,shapes,%
    matrix,shapes.symbols}


% Define block styles
\tikzstyle{block} = [rectangle, draw, fill=white!100, 
    text width=8em, text centered, rounded corners, minimum
    height=2em]
\tikzstyle{block2} = [rectangle, draw, fill=gray!20, 
    text width=8em, text centered, rounded corners, minimum height=2em]
\tikzstyle{line} = [draw, -latex']




%\usepackage{cite}


%%%%%%%%%%%%%%%%%%%%%%%% Some useful packages %%%%%%%%%%%%%%%%%%%%%%%%%%%%%%%
%%%%%%%%%%%%%%%%%%%%%%%% See related documentation %%%%%%%%%%%%%%%%%%%%%%%%%%
\usepackage{amsmath} % popular packages from Am. Math. Soc. Please use the 
\usepackage{amssymb} % related math environments (split, subequation, cases,
\usepackage{amsfonts}% multline, etc.)
%\usepackage{bm}      % Bold Math package, defines the command \bf{}
%\usepackage{paralist}% extended list environments
%%subfig.sty is the modern replacement for subfigure.sty. However, subfig.sty 
%%requires and automatically loads caption.sty which overrides class handling 
%%of captions. To prevent this problem, preload caption.sty with caption=false 
%\usepackage[caption=false]{caption}
%\usepackage[font=footnotesize]{subfig}


%user defined variables
\def\papertitle{Query-by-Multiple-Examples: Content-Based Search in
  Computer-Assisted Sound-Based Musical Composition}
\def\firstauthor{Tiago Fernandes Tavares}
\def\secondauthor{J\^{o}natas Manzolli}
\def\thirdauthor{Third author}

% adds the automatic
% Saves a lot of ouptut space in PDF... after conversion with the distiller
% Delete if you cannot get PS fonts working on your system.

% pdf-tex settings: detect automatically if run by latex or pdflatex
\newif\ifpdf
\ifx\pdfoutput\relax
\else
   \ifcase\pdfoutput
      \pdffalse
   \else
      \pdftrue
\fi

\ifpdf % compiling with pdflatex
  \usepackage[pdftex,
    pdftitle={\papertitle},
    pdfauthor={\firstauthor, \secondauthor},%, \thirdauthor},
    bookmarksnumbered, % use section numbers with bookmarks
    pdfstartview=XYZ % start with zoom=100% instead of full screen; 
                     % especially useful if working with a big screen :-)
   ]{hyperref}
  %\pdfcompresslevel=9

  %\usepackage[pdftex]{graphicx}
  % declare the path(s) where your graphic files are and their extensions so 
  %you won't have to specify these with every instance of \includegraphics
  %\graphicspath{{./figures/}}
  %\DeclareGraphicsExtensions{.pdf,.jpeg,.png}

  \usepackage[figure,table]{hypcap}

\else % compiling with latex
  \usepackage[dvips,
    bookmarksnumbered, % use section numbers with bookmarks
    pdfstartview=XYZ % start with zoom=100% instead of full screen
  ]{hyperref}  % hyperrefs are active in the pdf file after conversion

  \usepackage[dvips]{epsfig,graphicx}
  % declare the path(s) where your graphic files are and their extensions so 
  %you won't have to specify these with every instance of \includegraphics
  \graphicspath{{./figures/}}
  \DeclareGraphicsExtensions{.eps}

  \usepackage[figure,table]{hypcap}
\fi

%setup the hyperref package - make the links black without a surrounding frame
\hypersetup{
    colorlinks,%
    citecolor=black,%
    filecolor=black,%
    linkcolor=black,%
    urlcolor=black
}


% Title.
% ------
\title{\papertitle}

% Authors
% Please note that submissions are NOT anonymous, therefore 
% authors' names have to be VISIBLE in your manuscript. 
%
% Single address
% To use with only one author or several with the same address
% ---------------
\oneauthor
   {\firstauthor, \secondauthor} {Interdisciplinary Nucleus of Sound
     Communication\\Music Department --
     Institute of Arts\\University of Campinas\\Campinas - SP - Brazil \\ %
     {\tt \href{mailto:tiago@nics.unicamp.br}{tiago@nics.unicamp.br}}}

%Two addresses
%--------------
% \twoauthors
%   {\firstauthor} {Affiliation1 \\ %
%     {\tt \href{mailto:author1@smcnetwork.org}{author1@smcnetwork.org}}}
%   {\secondauthor} {Affiliation2 \\ %
%     {\tt \href{mailto:author2@smcnetwork.org}{author2@smcnetwork.org}}}

% Three addresses
% --------------
% \threeauthors
%   {\firstauthor} {Affiliation1 \\ %
%     {\tt \href{mailto:author1@smcnetwork.org}{author1@smcnetwork.org}}}
%   {\secondauthor} {Affiliation2 \\ %
%     {\tt \href{mailto:author2@smcnetwork.org}{author2@smcnetwork.org}}}
%   {\thirdauthor} { Affiliation3 \\ %
%     {\tt \href{mailto:author3@smcnetwork.org}{author3@smcnetwork.org}}}


% ***************************************** the document starts here ***************
\begin{document}
%
\capstartfalse
\maketitle
\capstarttrue
%
\begin{abstract}
We propose a search method, namely Query-by-Multiple-Examples, that is
able to search, within an audio sample 
database, for a particular sonic characteristic. The characteristic is
learned on-the-fly by means of multiple examples provided by a human
user, thus avoiding ambiguities due to manual labelling. We evaluate
four variations of the proposed method using ground truth provided
by three musicians. It is shown that, for queries based on sonic
characteristics, the query modelling process yields more correct
results than if several single-example queries were executed in
parallel using the same input data.
\end{abstract}
%

\section{Introduction}\label{sec:introduction}
Sound-based music is that in which the main discourse is
based on the evolution of sonic characteristics
\cite{Solomos2013}. This category 
comprises genres such as Electroacoustic, some kinds of Electronics,
Acousmatic, and Mixed Music. A frequent part of the composition 
process in these genres is recording sound samples from diverse
sources and using them as material -- either raw \cite{Schwarz2005},
processed \cite{Opie2003} or as
inspiration \cite{Nouno2009} -- for the construction of a piece.

Although it is common that a musician has a personal, well-known
sample database, the process of recording new samples may be
time-demanding. Collaborative databases allow a composer to
benefit from its peer's recording work, providing quick access to
many more sounds than it would be feasible to personally
record. We propose a search method that allows
semi-automatic search using personalized criteria, allowing composers
to find new, interesting 
sounds in sample databases that are too big for careful listening.

Many content-based search methods rely on semantic tags
\cite{Platt2002,Pauws2002,Kodama2005,Shao2009,Jensen2012,Bogdanov2013}
or other contextual data, like user ratings or popularity
\cite{Knees2013}, but these approaches are of limited use for
composers as they are often interested in sonic characteristics of an
audio sample, not the identification or perceived quality of the
recorded object. It is important to note that sonic
characteristics are often multi-dimensional, and composers are often
interested in nuances \cite{Moravec2005}, such as ``noisiness'' or ``brightness'', which may assume different meanings depending on the context
\cite{Sarkar2007}. Therefore, we 
propose a data-driven system as a solution for this problem.

In a data-driven search system \cite{Li2000,Lancieri2008}, sound samples are
mapped into a $\mathbb{R}^N$ vector space defined by low-level
features calculated 
from audio samples. These features aim at encoding the multiple
dimensions related to sound perception, which means perceptually similar sounds are
likely to be closer to each other \cite{Tzanetakis2002} according to
some distance measure \cite{Helen2010}. 
However, it is
impossible to know, from a single element, what perceptual
characteristics are desired by the composer and what are not
important; hence, the search system 
requires, as input, two or more examples so that it is possible to
know which dimensions should be considered or disregarded in the
search process.

We propose a novel search method, namely
\textit{Query-by-Multiple-Examples}, 
in which multiple examples are used to train a search machine
regarding what perceptual characteristics are desired 
by the user and what other characteristics may be disregarded. This
aims at providing a high level of customization in the search
criteria. Thus, the composer's perception is quickly modelled and
extended, allowing the retrieval of sound samples in a big database
according to personal criteria.

This paper is organized as follows.
%In Section \ref{sec:related}, we bring a brief review of related work. 
The implemented search methods
are described in Section \ref{sec:method}. The evaluation method, as
well as the results, are shown in Section
\ref{sec:evaluation}. Section \ref{sec:discussion} brings further
discussion and Section \ref{sec:conclusion} concludes the text.

% \section{Related Work}\label{sec:related}
% Content-based queries are used in audio and music databases in order
% to retrieve elements that fit particular perceptual or subjective
% criteria \cite{Li2000}. Essentially, this task is similar to
% classifying the elements of a database \cite{Tzanetakis2002} according
% to their similarity to an element provided as input (the query). For
% this kind of application, it is necessary to find a way to describe
% the elements of the database according to their content.

% One possible way of achieving this description is through contextual data
% \cite{Knees2013}, which comprises both metadata (keywords that describe
% database elements) and user-based measures (for example, user
% ratings). This approach, used by Platt et al. \cite{Platt2002}, Pauws et
% al. \cite{Pauws2002} and Kodama et al. \cite{Kodama2005}, assumes that
% two different users are likely to describe the same database element
% using similar words, which is the 
% case for large, cloud-based popular music databases. However, sonic
% textures are described by different people using 
% several different words \cite{Sarkar2007}, especially in the case of
% musicians that look for a particular timbral aspect
% \cite{Moravec2005}.

% This means that, for the purpose of timbre searches, a highly
% personalized search is desirable. For this application, metadata may
% be unreliable, and user-based measures prevent personalization of the
% search criteria. Therefore, a content-based approach \cite{Li2000} may
% prove more effective.

% Such approaches map database elements (i.e., audio samples) to a
% $\mathbb{R}^N$ vector space, in which samples that sound similar are likely to be
% placed in positions that are closer to each other
% \cite{Tzanetakis2002}. Then, a search algorithm is applied to find
% database elements that are more similar to the query
% \cite{Li2000}. By changing the mapping process and the search
% algorithm, several approaches for the recommendation problem have been
% proposed.


% Li and Khokhar \cite{Li2000} used a wavelet-based decomposition and a
% hierarchical clustering technique to retrieve animal sound samples
% from a database. Lancieri et al. \cite{Lancieri2008} proposed recommendation using
% features extracted directly from the acoustic signal. Similarly, Shao
% et al. \cite{Shao2009}, Bogadov et
% al. \cite{Bogdanov2011,Bogdanov2013} and Jensen et
% al. \cite{Jensen2012} proposed hybrid approaches, in
% which both acoustic features, metadata and user preferences are
% combined in an unique similarity 
% measure.

% Hel\'{e}n and Virtannen \cite{Helen2010} evaluated which
% similarity measures would give better results in Query-by-Example
% problems. The test used a dataset aimed at musical genre
% classification tasks. They show that estimating a Gaussian model from
% time-varying acoustic features, and then calculating the Euclidean
% distance between these models, performed best at the classification
% task.

% Schnitzer et al. \cite{Schnitzer2011} observed that, in content-based
% recommendation systems, an element could be close to many other
% elements, hence being recommended multiple times. This problem, namely
% ``the hub problem'', was solved by substituting the Euclidean distance
% for a measure called Mutual Proximity (MP). The Mutual Proximity is a
% measure of how close an element is to another element and, at the same
% time, how far it is to other elements of the dataset.

% In the Query-by-Multiple-Examples, several examples are used to train
% a search machine regarding what perceptual characteristic is desired
% by the user and what other characteristics may be disregarded. This
% way, we achieve a high level of customization in the search
% criteria. More details regarding the proposed system will be discussed
% in the next section.

\section{Proposed Method}\label{sec:method}
The proposed system is built as to merge two different sources of
information, as depicted in Figure \ref{fig:block-diagram}. The first
source is the composer, which provides audio examples of the sonic
characteristic that is desired to be found (a query). The second
source is the computer, which aims at extending, for all the database,
the criteria applied to the construction of the query.
\begin{figure}[ht!]
\centering
\begin{tikzpicture}[node distance = 4cm, auto]
    % Place nodes
    \node [block] (audio) {Audio Dataset};
    \node [block2, below of=audio, yshift=2.7cm] (composer) {Composer};
    \node [block, below of=composer, yshift=2.7cm] (perception) {Perception};
%    \node [block, below of=perception, yshift=2.7cm] (query) {Examples};

    \node [block2, right of=composer] (computer) {Computer};
    \node [block, below of=computer, yshift=2.7cm] (features) {Features};
    \node [block, below of=features, yshift=2.7cm] (model) {Model};
    \node [block, below of=model, yshift=2.7cm] (recommendation) {Recommendation};

    % Draw edges
    \path [line] (audio) -- (composer);
    \path [line] (audio) -| (computer);
    \path [line] (composer) -- (perception);
%    \path [line] (perception) -- (query);
    \path [line] (perception) |- node [xshift=1cm] {Query $\{e_j\}$}  (model);
    \path [line] (computer) -- (features);
    \path [line] (features) -- node {$\{d_i\} \in \mathbb{R}^N$} (model);
    \path [line] (model) -- (recommendation);
    \path [line,dashed] (recommendation) -| ([xshift=-0.5cm] composer.west) -- (composer);
\end{tikzpicture}
\label{fig:block-diagram}
\caption{Block diagram for the proposed system.}
\end{figure}

By combining the objective, vector representation of audio samples and
the subjective, perceptually-driven query, the system builds a
model for what it detects as the characteristic sought by the
composer. This model is, then, extrapolated, so that other audio
samples that correspond to that characteristic, are found. Then, the
system yields a recommendation, which can be used by the composer.

In our work, the calculation of features (yielding representations in
the $\mathbb{R}^N$ vector space) follows the same general structure
used in previous work \cite{Tzanetakis2002,Helen2010}, as described in
Section \ref{sec:features}. However, we experimented several different
methods for modelling queries. This will be described in details in
Section \ref{sec:queries}.

\subsection{Feature calculations}\label{sec:features}
The process of obtaining a vector representation in the feature
space begins with calculating a framewise Short-Time Fourier Transform, using 23ms
frames, with a 50\% overlap ratio, multiplying each frame by a Hanning
window and then calculating the absolute value $X_q[k]$ of the DFT of
each frame. For each frame $q$, a set of acoustic features are calculated, as described in
Table \ref{tab:features}. Also, the first and second-order
differentials of each features are calculated.
\begin{table*}
\centering
\caption{Brief description of acoustic features}
\begin{tabular}{|c|c|} \hline
Feature & Description \\ \hline \hline
Energy & Sum of the squared values of $X_q[k]$. Indicates how loud the frame is. \\
Spectral centroid & Centroid of $X_q[k]$. Correlates with the brightness
of the frame. \\
Spectral roll-off & Frequency above which there is less than 15\% of
the energy of a frame. \\
Spectral flatness & Indicates how close the frame is to white
noise. \\
Mel-Frequency Cepstral Coefficients & Vector representation
of audio textures, inspired in cochlea models. \\ \hline
\end{tabular}
\label{tab:features}
\end{table*}

For each feature and its differentials, we calculate a set of
statistics. This set comprises mean and variance, which give an idea
of the general behaviour of these features. We also obtain the slope
(considering a linear regression) and the value and time location of
the maximum and minimum values, to depict the
evolution of features over time.

This process associates each audio sample to a descriptive vector,
which we expect to encode its perceptual characteristics. As it will
be seen, there are many ways to model multiple-example queries. This
will be discussed in the next section.

\subsection{Models for queries}\label{sec:queries}
The modelling process for queries aims at detecting relevant sonic
characteristics as described by the composer using examples
$e_j$. This process assumes that these characteristics are encoded within the
objective $\mathbb{R}^N$ feature space defined by the calculated
features (as described in Section \ref{sec:features}). The model gives
a score to each element $d_i$ of the database, and infers that the element
with the highest score also presents the characteristic desired by the
composer.

We evaluated several methods for obtaining the model. This was done
because there is no particular prior reason to assume that a model is
better than another, hence a detailed evaluation is necessary. Each
method has its own rationale, which will be described below.

The first method, \textbf{Minimum Distance} (MD), gives an element a score
which is the inverse of the minimum Euclidean distance between itself
and any element from the query, as depicted in Expression
\ref{eq:md}. It is
equivalent to performing a few queries-by-example (using single
examples) in parallel, and then selecting the best results. Hence, it
may not be considered as a valid method for Query-by-Multiple-Examples.
\begin{equation}
{\mbox{MD}}(d_i) = 1/(\min_j \Vert d_i - e_j \Vert).
\label{eq:md}
\end{equation}

The second method, \textbf{Minimum Mutual Distance} (MMD), was
inspired by work by Schnitzer et. al \cite{Schnitzer2011}, which
observes that an element that belong to a cluster must not only be close
to the cluster but also distant from other clusters. Hence, it scores
each sample from the database with the minimum distance between itself
and an element from the query minus the minimum distance between
itself and an element in the database, as described in Expression
\ref{eq:mmd}. Although this method is more complex than MD, it also
does not perform a Query-by-Multiple-Examples, but multiple queries-by-single-example
in parallel.
\begin{equation}
{\mbox{MMD}}(d_i) = 1/(\min_{j} \Vert d_i - e_j \Vert - \min_{k} \Vert d_i - d_k \Vert).
\label{eq:mmd}
\end{equation}

Third, we consider the \textbf{Naive Bayes} (NB) approach. In this
method, the elements of the 
query are used to estimate the mean $\mu_n$ and variance $\sigma_n$ of
a gaussian model for each dimension, which is assumed to be
independent from the others. Thus, the score of an element of the database is given
by:
\begin{equation}
\mbox{NB}(d_i) = \prod_{n=1}^N g(d_{i,n}, \mu_n, \sigma_n),
\end{equation}
where $g(x, \mu, \sigma) = (1/\sigma \sqrt{2 \pi}) \exp(-(\mu - x)^2 / 2\sigma ^2)$.

%The Gaussian model applied in the NB approach assumes that the points
%of interest are located in a convex region in the $\mathbb{R}^N$
%vector space representation.

The NB approach assumes that dimensions spanned by features are
orthogonal. There is no
evidence that this condition is true, therefore we applied Principal
Component Analysis (PCA) to obtain an orthogonal projection
$\boldsymbol B$ of the query with a minimal approximation error. We
expect that this projection will be a better representation for the
composer's perception than the raw feature set itself.

The projection is made using $M-1$ vectors, where $M$ is the number of
elements in the query, because this is the maximum rank of the
projection provided by PCA. The projection is calculated using only
elements from the query, and then applied over the whole
database. Then, the naive Bayes approach is used normally as described
above, hence the method is named \textbf{PCA-Bayes} (PB).

Hence, four different modelling methods were applied. Two of them are simple
applications of simple query-by-example schemas, whereas the other two
use the correlations within the query to build a different model. For
comparison purposes, a random recommender (recommending a random
element from the database) was also used in the evaluation set, which
will be described in detail in the next section.

\section{Evaluation and Results}\label{sec:evaluation}
The evaluation process aimed at detecting whether the system is able
to retrieve audio samples from the database as if the composer was
searching for it.

To reproduce this scenario, we interviewed three
composers, all of them graduate students from the Music
Department. After a brief talk about their composition processes, we
asked them to group pre-defined audio samples (from a set of 51
elements, around 10s long, extracted from a personal music collection)
into subsets that made sense for them, and, if possible, explain what
criteria was used for grouping. Their criteria was significantly
different, as it is discussed below.

Composer \textbf{C1} stated that processed samples were used as
material in the piece composition process in order to reach a
particular sound characteristic. The
presented sets were predominantly grouped using characteristics linked
to auditory aspects of each sample. Typical grouping criteria were
labeled \textit{dry/dark timbre}, \textit{static harmonic sound},
\textit{brightness and compression} and \textit{glissando}. 

Composer \textbf{C2}'s composition process uses the semantic
values of the audio samples, in addition to their sound. The grouping
process considered semantic-valued characteristics, that is, the
context in which each sample was obtained. In this case, grouping
criteria were of higher level, such as \textit{celtic}, \textit{drums}
and \textit{prepared piano}.

Composer \textbf{C3} preferred to use sound samples as source for
granular synthesis processes. Grouping criteria was based on auditory
characteristics of samples, as well as general semantics. Among the
grouping criteria, it was possible to find \textit{vocal},
\textit{regular}, \textit{synthesizer},
\textit{orchestral} and \textit{attack modes}.

The subsets presented by each composer were assumed as ground-truth,
that is, a query containing some elements of each subset should find
the remaining elements. Several queries were made from each group,
considering different numbers of elements. The queries made from the
groups of each composer were considered separately, so that their
different reasoning towards the search process could be analyzed.

For each query, the system was asked to retrieve three samples from
the database. A retrieved sample is considered correct if it belongs to the
subset from which the query was made. Then, we calculated two accuracy
measures, \textit{Acc1} and \textit{Acc2}, defined as:
\begin{equation}
\mbox{Acc1} = \frac{\mbox{\# retrievals with at least 1 correct
    sample}}{\mbox{\# total queries} }.
\end{equation}
\begin{equation}
\mbox{Acc2} = \frac{\mbox{\# correctly retrieved samples}}{ \mbox{\# total retrieved samples} }
\end{equation}

\textit{Acc1} measures the probability that a query will retrieve at
least one useful sample, which is desirable because it means that the
search space has been narrowed. \textit{Acc2} measures the probability
that a retrieved sample is useful, which is also a desirable
characteristic of the system. It is important to note that
\textit{Acc1} is higher when the system avoids false negatives,
whereas \textit{Acc2} is higher when the system presents fewer false
positives.

We tested the system using all query modelling methods discussed in
Section \ref{sec:queries}. The results for the datasets related to
composers C1, C2 and C3 were considered separately. \textit{Acc1} and
\textit{Acc2} for each test case are shown, respectively, in Figures
\ref{fig:res_acc1} and \ref{fig:res_acc2}.
\begin{figure}[ht!]
\centering
\caption{Results using Acc1.} 
\begin{tikzpicture}
\pgfplotstableread{ % Read the data into a table macro
Label   MD  MMD  NB PB  
C1      38   44   42  52 
C2      42   51   51  43
C3      61   72   67  76
}\datatable

\begin{axis}[
  width=0.45\textwidth,
  height=5cm,
	xtick=data,
  ymin=0,
  ymax=100,
  ylabel=Acc1 (\%),
	xlabel=Composers,
	enlarge x limits = 0.20,
	legend style={legend columns=-1},
	ybar,
	bar width=10pt,
  xticklabels from table={\datatable}{Label}
]

\addplot [fill=white] table [x expr=\coordindex, y=MD] {\datatable};    % Plot the "First" column against the data index
\addplot [fill=gray!30]table [x expr=\coordindex, y=MMD] {\datatable};
\addplot [fill=gray!90] table [x expr=\coordindex, y=NB] {\datatable};
\addplot [fill=black] table [x expr=\coordindex, y=PB] {\datatable};
% \addplot 
% 	coordinates {(MD,0.38) (MMD,0.44)
% 		 (NB,0.42) (PB,0.52)};

% \addplot 
% 	coordinates {(MD,0.42) (MMD,0.51)
% 		 (NB,0.51) (PB,0.43)};

% \addplot 
% 	coordinates {(MD,0.61) (MMD,0.72)
% 		 (NB,0.67) (PB,0.76)};


\legend{MD,MMD,NB,PB}
\end{axis}
\end{tikzpicture}
\label{fig:res_acc1}
\end{figure}
\begin{figure}[ht!]
\centering
\caption{Results using Acc2.} 
\begin{tikzpicture}
\pgfplotstableread{ % Read the data into a table macro
Label   MD  MMD  NB PB  
C1      17   17   17  21 
C2      17   21   23  22
C3      27   34   33   40 
}\datatable

\begin{axis}[
  width=0.45\textwidth,
  height=5cm,
	xtick=data,
  ymin=0,
  ymax=100,
  ylabel=Acc2 (\%),
	xlabel=Composers,
	enlarge x limits = 0.20,
	legend style={legend columns=-1},
	ybar,
	bar width=10pt,
  xticklabels from table={\datatable}{Label}
]

\addplot [fill=white] table [x expr=\coordindex, y=MD] {\datatable};    % Plot the "First" column against the data index
\addplot [fill=gray!30]table [x expr=\coordindex, y=MMD] {\datatable};
\addplot [fill=gray!90] table [x expr=\coordindex, y=NB] {\datatable};
\addplot [fill=black] table [x expr=\coordindex, y=PB] {\datatable};
% \addplot  
% 	coordinates {(MD,0.38) (MMD,0.44)
% 		 (NB,0.42) (PB,0.52)};

% \addplot 
% 	coordinates {(MD,0.42) (MMD,0.51)
% 		 (NB,0.51) (PB,0.43)};

% \addplot 
% 	coordinates {(MD,0.61) (MMD,0.72)
% 		 (NB,0.67) (PB,0.76)};


\legend{MD,MMD,NB,PB}
\end{axis}
\end{tikzpicture}
\label{fig:res_acc2}
\end{figure}

As it can be seen, the MD method performed worse for all of the test
cases. Nevertheless, its results are comparable to those yielded by
the other methods, i.e., it is not possible to claim that the
difference is huge. Hence, MD may be considered a baseline method for
further discussion.

Another key result is that MMD always performs better than MD. This
shows that its assumption -- that an element that belongs to a set
must not only be close to the set, but also distant from the other
sets \cite{Schnitzer2011} -- is useful for our purposes. More
interesting results, however, derived from the application of the NB
and PB methods.

In the case of C1 and C3, it is clear that PB outperforms all other
methods, considering either \textit{Acc1} or \textit{Acc2}. However,
in the case of C2, PB is outperformed by both NB and MMD for
\textit{Acc1} and has a similar performance considering
\textit{Acc2}. This may be due to the grouping process performed by
composer C2.

Composer C2 applied a predominantly semantic grouping of
elements, that is, elements that do not sound alike, but are
culturally related (for example: the sounds of Celtic fiddles and of
Irish tap dancers) were grouped together. This means that the feature
space -- which describes auditory characteristics -- was divided in
several clusters with useful data. This caused the MMD method to
present a better performance.

Also, NB performed better than PB for this case, which shows
that the original feature set defined better local maxima to the score
function than the reduced-dimension orthogonal set. Although
orthogonality is a desirable trait, it is important to note that a
high-dimensional space has a greater chance to have at least one
dimension in which any points, chosen at random, are positioned in a
linearly separable convex hull. However, if more dimensions were used in
the PCA reduction, they would be linearly dependent of the previous ones,
which means another method for dimension reduction would be required.

The next section conducts further discussion on these results.

\section{Discussion}\label{sec:discussion}
One interesting point shown by the results is that they are highly
dependent on the criteria used by the composer for
classification. In our tests, Acc1 varied from 50\% to 75\%, and Acc2
from 20\% to 40\%. This is a great relative step, which has to be
considered when performing future evaluations.

Although the system was evaluated using objective measures, it is
important to note that it is a recommendation system, which will
interact with users. Hence, it is necessary to conduct further tests
to detect whether 
the results provided by the system are useful for the composer,
despite of not being expected \textit{a priori}. These results may
show if the system is able to recommend useful samples (maybe some
sample that may be used, but the composer would not have thought of
about alone), thus allowing generalization towards a bigger database.

The results obtained above show that each mindset for sample grouping
-- auditory or semantic -- can be better modelled by a different
algorithm: auditory-based criteria are well suited for the PB method,
whereas semantic-based are better modelled using MMD. This happens
because the dimensions spanned by acoustic describe auditory
characteristics, which means semantic information is only present as
an underlying function of the acoustic features. Possible ways to deal
with this situation may involve other dimensionality-reduction
techniques, such as Independent Component Analysis (ICA) or non-linear
PCA.

Although MMD -- which corresponds to multiple
queries by single-example -- outperforms NB and PB for the queries
corresponding to composer C2, it is important to note that the system
was built aiming at detecting sonic characteristics, rather than
semantics. For the queries corresponding to composers C1 and C3, which
follow the idea of describing sounds without considering semantics,
PB -- a method that clearly takes advantage of the correlations within
the multiple example query -- outperforms all others. Therefore, the
results show that the proposed system, using PB, provided a meaningful
contribution towards the problem of searching for sound
characteristics within a database.

Next section presents conclusive remarks.

\section{Conclusion}\label{sec:conclusion}
We presented a search method, namely Query-By-Multiple-Examples. It 
receives as input a few audio samples that examples of a particular
sound characteristic yields other samples, from a database, that also
present that characteristic. The method is aimed at extending the
search possibilities of composers in the context of
sound-based music, that is, music based on the evolution of sonic
characteristics.

The method is based on mapping all audio samples from a database into
a vector space using low-level acoustic features. Queries are received
from the user and modelled, yielding a score for each element in the
database. We tested four different methods for modelling the query:
minimum distance and minimum mutual distance (corresponding to several
queries-by-single-example executed in parallel), and naive Bayes and
PCA-Bayes (corresponding to query-by-multiple-example).

We evaluated all variations using ground-truth queries, defined by
three different musicians. They provided very different
proposals for the grouping of similar audio samples, according to
their typical composition process. It has been shown that considering
the correlations within the provided inputs improves the search
accuracy for auditory-inspired queries.

%It has been shown that
%auditory-related queries are well modelled by 
%the Bayesian methods, whereas element-wise distances (minimum distance
%and minimum mutual distance) are more
%suitable for semantic-based queries.

The obtained results, however, do not account for the interaction
between composers and the computer, which is an important part of the
composition process. Thus, it is necessary to
evaluate whether the wrong results yielded by the system are useful
suggestions (despite of being unexpected) or if they are just plain
wrong, and, more than that, how the system would behave in an unknown
database. This points a clear direction for future work.

\begin{acknowledgments}
The authors thank FAPESP (proc. 2013/17329-5) for funding this research.
\end{acknowledgments} 

%%%%%%%%%%%%%%%%%%%%%%%%%%%%%%%%%%%%%%%%%%%%%%%%%%%%%%%%%%%%%%%%%%%%%%%%%%%%%
%bibliography here
\bibliography{recommendation}


\end{document}
